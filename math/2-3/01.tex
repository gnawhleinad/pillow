\documentclass[12pt]{article}
\usepackage{fancyhdr}
\usepackage{amsmath,amsfonts,enumerate}
\usepackage{color,graphicx}
\pagestyle{fancy}

\cfoot{\thepage}
\renewcommand{\headrulewidth}{0.0pt}

\begin{document}

\paragraph{1.}
Fill in the missing number in each box. \\
\\
one-hundred forty-two = \framebox(30,15){} \textbf{hundreds} + \framebox(30,15){} \textbf{tens} + \framebox(30,15){} \textbf{ones}
\\
\\
four-hundred twenty-one = \framebox(30,15){} \textbf{hundreds} + \framebox(30,15){} \textbf{tens} + \framebox(30,15){} \textbf{ones}
\\
\\
two-hundred forty-one = \framebox(30,15){} \textbf{hundreds} + \framebox(30,15){} \textbf{tens} + \framebox(30,15){} \textbf{ones}
\\
\\
two-hundred fourteen = \framebox(30,15){} \textbf{hundreds} + \framebox(30,15){} \textbf{tens} + \framebox(30,15){} \textbf{ones}
\\
\\
two-hundred twenty-two = \framebox(30,15){} \textbf{hundreds} + \framebox(30,15){} \textbf{tens} + \framebox(30,15){} \textbf{ones}
\\
\\
six-hundred thirteen = \framebox(30,15){} \textbf{hundreds} + \framebox(30,15){} \textbf{tens} + \framebox(30,15){} \textbf{ones}
\\
\\
nine-hundred seventy = \framebox(30,15){} \textbf{hundreds} + \framebox(30,15){} \textbf{tens} + \framebox(30,15){} \textbf{ones}
\\
\\
eight-hundred seventeen = \framebox(30,15){} \textbf{hundreds} + \framebox(30,15){} \textbf{tens} + \framebox(30,15){} \textbf{ones}
\\
\\
241 = \framebox(30,15){} \textbf{hundreds} + \framebox(30,15){} \textbf{tens} + \framebox(30,15){} \textbf{ones}
\\
\\
124 = \framebox(30,15){} \textbf{hundreds} + \framebox(30,15){} \textbf{tens} + \framebox(30,15){} \textbf{ones}
\\
\\
142 = \framebox(30,15){} \textbf{hundreds} + \framebox(30,15){} \textbf{tens} + \framebox(30,15){} \textbf{ones}
\\
\\
412 = \framebox(30,15){} \textbf{hundreds} + \framebox(30,15){} \textbf{tens} + \framebox(30,15){} \textbf{ones}
\\
\\
222 = \framebox(30,15){} \textbf{hundreds} + \framebox(30,15){} \textbf{tens} + \framebox(30,15){} \textbf{ones}
\\
\\
316 = \framebox(30,15){} \textbf{hundreds} + \framebox(30,15){} \textbf{tens} + \framebox(30,15){} \textbf{ones}
\\
\\
79 = \framebox(30,15){} \textbf{hundreds} + \framebox(30,15){} \textbf{tens} + \framebox(30,15){} \textbf{ones}
\\
\\
718 = \framebox(30,15){} \textbf{hundreds} + \framebox(30,15){} \textbf{tens} + \framebox(30,15){} \textbf{ones}

\bigskip
\bigskip
\bigskip
\newpage

\paragraph{2.}
Fill in the missing number in each box. \\
\\
3 \textbf{hundreds} = \framebox(30,15){} \textbf{ones}
\\
\\
\framebox(30,15){} \textbf{hundreds} = 800 \textbf{ones}
\\
\\
1 \textbf{hundreds} = \framebox(30,15){} \textbf{ones}
\\
\\
\framebox(30,15){} \textbf{hundreds} = 200 \textbf{ones}
\\
\\
\framebox(30,15){} \textbf{hundreds} = 700 \textbf{ones}
\\
\\
7 \textbf{hundreds} = \framebox(30,15){} \textbf{ones}
\\
\\
\framebox(30,15){} \textbf{hundreds} = 400 \textbf{ones}
\\
\\
\framebox(30,15){} \textbf{hundreds} = 500 \textbf{ones}
\\
\\
2 \textbf{hundreds} = \framebox(30,15){} \textbf{ones}

\bigskip
\bigskip
\bigskip
\newpage

\paragraph{3.}
Count \textbf{forward} by \textbf{fives} from each number. \\
\\
40, \framebox(30,15){}, \framebox(30,15){}, \framebox(30,15){}, \framebox(30,15){}, \framebox(30,15){}
\\
\\
75, \framebox(30,15){}, \framebox(30,15){}, \framebox(30,15){}, \framebox(30,15){}, \framebox(30,15){}
\\
\\
5, \framebox(30,15){}, \framebox(30,15){}, \framebox(30,15){}, \framebox(30,15){}, \framebox(30,15){}

\bigskip

\paragraph{4.}
Count \textbf{backward} by \textbf{two} from each number. \\
\\
40, \framebox(30,15){}, \framebox(30,15){}, \framebox(30,15){}, \framebox(30,15){}, \framebox(30,15){}
\\
\\
75, \framebox(30,15){}, \framebox(30,15){}, \framebox(30,15){}, \framebox(30,15){}, \framebox(30,15){}
\\
\\
99, \framebox(30,15){}, \framebox(30,15){}, \framebox(30,15){}, \framebox(30,15){}, \framebox(30,15){}

\bigskip

\paragraph{5.}
Count \textbf{forward} by \textbf{two} from each number. \\
\\
40, \framebox(30,15){}, \framebox(30,15){}, \framebox(30,15){}, \framebox(30,15){}, \framebox(30,15){}
\\
\\
75, \framebox(30,15){}, \framebox(30,15){}, \framebox(30,15){}, \framebox(30,15){}, \framebox(30,15){}
\\
\\
5, \framebox(30,15){}, \framebox(30,15){}, \framebox(30,15){}, \framebox(30,15){}, \framebox(30,15){}

\bigskip

\paragraph{6.}
Count \textbf{forward} by \textbf{hundred} from each number. \\
\\
300, \framebox(30,15){}, \framebox(30,15){}, \framebox(30,15){}, \framebox(30,15){}, \framebox(30,15){}
\\
\\
121, \framebox(30,15){}, \framebox(30,15){}, \framebox(30,15){}, \framebox(30,15){}, \framebox(30,15){}
\\
\\
229, \framebox(30,15){}, \framebox(30,15){}, \framebox(30,15){}, \framebox(30,15){}, \framebox(30,15){}

\newpage

\paragraph{7.}
Circle all the lines that make \textbf{1}. \\
\\
9 - 0
\\
\\
9 - 1
\\
\\
9 - 7
\\
\\
9 - 8
\\
\\
6 - 5
\\
\\
3 - 4
\\
\\
2 - 4
\\
\\
2 - 1
\\
\\
10 - 9 
\\
\\
18 - 12
\\
\\
1 - 0
\\
\\
5 - 3 
\\
\\
7 - 2
\\
\\
7 - 5
\\
\\
7 - 6
\\
\\
6 - 7
\\
\\
17 - 6
\\
\\
17 - 16

\newpage

\paragraph{8.}
Circle all the lines that make \textbf{6}. \\
\\
9 + 0
\\
\\
1 - 1
\\
\\
9 - 3 
\\
\\
9 - 8
\\
\\
6 - 2
\\
\\
3 + 3
\\
\\
2 + 4
\\
\\
2 - 1
\\
\\
10 - 4
\\
\\
18 - 12
\\
\\
1 + 0
\\
\\
5 - 3 
\\
\\
7 + 2
\\
\\
7 + 5
\\
\\
7 - 6
\\
\\
6 + 2 
\\
\\
9 - 3
\\
\\
13 - 6

\newpage

\paragraph{9.}
Fill in the missing number in each box. \\
\\
18 = \framebox(30,15){} + 9
\\
\\
8 = 12 - \framebox(30,15){}
\\
\\
10 = \framebox(30,15){} + 9
\\
\\
9 = 1 + \framebox(30,15){}
\\
\\
7 = 12 - \framebox(30,15){}
\\
\\
12 = \framebox(30,15){} - 1
\\
\\
1 = 2 - \framebox(30,15){}
\\
\\
17 = \framebox(30,15){} + 9
\\
\\
1 = 12 - \framebox(30,15){}
\\
\\
1 = \framebox(30,15){} + 1
\\
\\
4 = 9 - \framebox(30,15){}
\\
\\
3 = 8 - \framebox(30,15){}
\\
\\
16 = \framebox(30,15){} + 9
\\
\\
2 = 7 - \framebox(30,15){}
\\
\\

\newpage

\paragraph{10.}
Add. \\
\bigskip

\begin{tabular}{c@{\,}c@{\,}c}
  & 2 & 3 \\
+ &   & 4 \\
\hline
\end{tabular}

\bigskip
\bigskip
\bigskip

\begin{tabular}{c@{\,}c@{\,}c}
  & 1 & 9 \\
+ &   & 6 \\
\hline
\end{tabular}

\bigskip
\bigskip
\bigskip

\begin{tabular}{c@{\,}c@{\,}c}
  & 5 & 5 \\
+ &   & 2 \\
\hline
\end{tabular}

\bigskip
\bigskip
\bigskip

\begin{tabular}{c@{\,}c@{\,}c}
  & 3 & 0 \\
+ &   & 6 \\
\hline
\end{tabular}

\bigskip
\bigskip
\bigskip

\begin{tabular}{c@{\,}c@{\,}c}
  & 6 & 7 \\
+ &   & 7 \\
\hline
\end{tabular}

\bigskip
\bigskip
\bigskip

\begin{tabular}{c@{\,}c@{\,}c}
  & 3 & 3 \\
+ &   & 3 \\
\hline
\end{tabular}

\bigskip
\bigskip
\bigskip

\begin{tabular}{c@{\,}c@{\,}c}
  & 8 & 9 \\
+ &   & 2 \\
\hline
\end{tabular}

\bigskip
\bigskip
\bigskip

\begin{tabular}{c@{\,}c@{\,}c}
  & 7 & 3 \\
+ &   & 9 \\
\hline
\end{tabular}

\end{document}
