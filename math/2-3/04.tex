\documentclass[12pt]{article}
\usepackage{fancyhdr}
\usepackage{amsmath,amsfonts,enumerate}
\usepackage{color,graphicx}
\pagestyle{fancy}

\cfoot{\thepage}
\renewcommand{\headrulewidth}{0.0pt}

\begin{document}

\paragraph{Lesson.}
The goal of this lesson is to learn how to solve problems like the one below: \\
\\
18 = \framebox(30,15){} + 9
\\
\\
By re-arranging the equations so that the \framebox(30,15){} is by itself, the problem becomes much easier to solve: \\
\\
18 - 9 = \framebox(30,15){}
\\
\\
So, if we go back to the original problem: \\
\\
18 = \framebox(30,15){} + 9
\\
\\
We can \textbf{subtract 9} to the left and right side of the equation so that \framebox(30,15){} can be alone: \\
\\
18 \textbf{- 9} = \framebox(30,15){} + 9 \textbf{- 9} \\
\\
18 - 9 = \framebox(30,15){} \\
\\
9 = \framebox(30,15){}

\newpage


\newpage

\paragraph{1.}
Fill in the missing number in each box. Show your work by re-arranging the equation. \\
\\
19 = \framebox(30,15){} + 10
\\
\\
9 = 13 - \framebox(30,15){}
\\
\\
11 = \framebox(30,15){} + 9
\\
\\
9 = 2 + \framebox(30,15){}
\\
\\
8 = 13 - \framebox(30,15){}
\\
\\
13 = \framebox(30,15){} - 2
\\
\\
2 = 3 - \framebox(30,15){}
\\
\\
19 = \framebox(30,15){} + 9
\\
\\
0 = 13 - \framebox(30,15){}
\\
\\
3 = \framebox(30,15){} + 2
\\
\\
5 = 9 - \framebox(30,15){}
\\
\\
4 = 8 - \framebox(30,15){}
\\
\\
17 = \framebox(30,15){} + 10
\\
\\
3 = 8 - \framebox(30,15){}

\paragraph{2.}
Multiply. \\
\\
13 * 2 = \framebox(30,15){}
\\
\\
14 * 3 = \framebox(30,15){}
\\
\\
15 * 4 = \framebox(30,15){}
\\
\\
16 * 5 = \framebox(30,15){}
\\
\\
13 * 8 = \framebox(30,15){}
\\
\\
12 * 7 = \framebox(30,15){}
\\
\\
12 * 5 = \framebox(30,15){}
\\
\\
13 * 6 = \framebox(30,15){}
\\
\\
15 * 9 = \framebox(30,15){}
\\
\\
16 * 10 = \framebox(30,15){}
\\
\\
18 * 3 = \framebox(30,15){}
\\
\\
19 * 4 = \framebox(30,15){}
\\
\\
19 * 9 = \framebox(30,15){}
\\
\\
20 * 10 = \framebox(30,15){}
\\
\\
20 * 1 = \framebox(30,15){}
\\
\\
21 * 2 = \framebox(30,15){}
\\
\\
13 * 4 = \framebox(30,15){}

\end{document}
