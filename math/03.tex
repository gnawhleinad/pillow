\documentclass[12pt]{article}
\usepackage{fancyhdr}
\usepackage{amsmath,amsfonts,enumerate}
\usepackage{color,graphicx}
\pagestyle{fancy}

\cfoot{\thepage}
\renewcommand{\headrulewidth}{0.0pt}

\begin{document}

\paragraph{Lesson.}
The following is an equation: \\
\\
3 = 3 \\
\\
The equation says 3 is equal to 3 which is obviously true. This equation is correct. However, consider the following equation: \\
4 = 3 \\
\\
This equation says 4 is equal to 3. This equation is false and incorrect.
\\
\\
The following is also an equation: \\
\\
1 + 2 = 3 \\
\\
This equation is true because we know 1 + 2 is equal to 3.

\newpage

\paragraph{1.} 
Circle all equations that are true. \\
\\
1 = 1
\\
\\
1 = 0
\\
\\
1 - 1 = 0
\\
\\
2 - 1 = 0
\\
\\
5 - 3 = 3
\\
\\
6 + 7 = 14
\\
\\
5 - 3 = 2
\\
\\
8 + 7 = 15
\\
\\
9 + 3 = 13
\\
\\
4 + 13 = 18
\\
\\
18 - 13 = 5
\\
\\
5 - 4 = 2
\\
\\
2 + 4 = 7
\\
\\
7 + 9 = 16
\\
\\
22 - 18 = 3
\\
\\
21 - 17 = 3
\\
\\
4 - 4 = 0
\\
\\
5 - 4 = 2

\newpage

\paragraph{Lesson.}
There are times when we want to make changes to an equation but continue to keep it true. In those times, we must make the same change to the left and right side of the equation. If we don't, then the equation will no longer be true. For example, if we add 1 only to the left, \\
\\
1 + 2 \textbf{+ 1} = 3 \\
4 = 3 \\
\\
Or, if we only add 1 to the right, \\
\\
1 + 2 = 3 \textbf{+ 1} \\
3 = 4 \\
\\
If we add them to both sides of the expression, then the equation is still true. \\
\\
1 + 2 \textbf{+ 1} = 3 \textbf{+ 1} \\
4 = 4

\newpage

\paragraph{2.}
Fill in the missing number in each box. \\
\\
2 + 4 = 6 \\
\\
6 - \framebox(30,15){} = 2 \\
\\
\\
\\
9 - 1 = 8 \\
\\
8 + \framebox(30,15){} = 9 \\
\\
\\
\\
15 - 7 = 8 \\
\\
8 + \framebox(30,15){} = 15 \\
\\
\\
\\
5 = 3 + 2 \\
\\
5 - \framebox(30,15){} = 2 \\
\\
\\
\\
11 = 13 - 2 \\
\\
11 + \framebox(30,15){} = 13 \\
\\
\\
\\
21 = 15 + 6 \\
\\
21 - \framebox(30,15){} = 15 \\

\newpage


\paragraph{Lesson.}
The goal of this lesson is to learn how to solve problems like the one below: \\
\\
18 = \framebox(30,15){} + 9
\\
\\
By re-arranging the equations so that the \framebox(30,15){} is by itself, the problem becomes much easier to solve: \\
\\
18 - 9 = \framebox(30,15){}
\\
\\
So, if we go back to the original problem: \\
\\
18 = \framebox(30,15){} + 9
\\
\\
We can \textbf{subtract 9} to the left and right side of the equation so that \framebox(30,15){} can be alone: \\
\\
18 \textbf{- 9} = \framebox(30,15){} + 9 \textbf{- 9} \\
\\
18 - 9 = \framebox(30,15){} \\
\\
9 = \framebox(30,15){}

\newpage


\newpage

\paragraph{3.}
Fill in the missing number in each box. Show your work by re-arranging the equation. \\
\\
18 = \framebox(30,15){} + 9
\\
\\
8 = 12 - \framebox(30,15){}
\\
\\
10 = \framebox(30,15){} + 9
\\
\\
9 = 1 + \framebox(30,15){}
\\
\\
7 = 12 - \framebox(30,15){}
\\
\\
12 = \framebox(30,15){} - 1
\\
\\
1 = 2 - \framebox(30,15){}
\\
\\
17 = \framebox(30,15){} + 9
\\
\\
1 = 12 - \framebox(30,15){}
\\
\\
1 = \framebox(30,15){} + 1
\\
\\
4 = 9 - \framebox(30,15){}
\\
\\
3 = 8 - \framebox(30,15){}
\\
\\
16 = \framebox(30,15){} + 9
\\
\\
2 = 7 - \framebox(30,15){}
\\
\\

\end{document}
